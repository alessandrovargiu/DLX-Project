%%%%%%%%%%%%%%%%%%%%%%%%%%%%%%%%%%%%%%%%%%%%%%%%%%%%
% This will help you in writing your homebook
% Remember that the character % is a comment in latex
%
% 
\chapter{Abstract}
\label{Abstract}

%%%%%%%%%%%%%%%%%%%%%%%%%%%%%%%%%%%%%%%%%%%%%%%%%%%%%%%%%%%
% you can organize a chapter using sections -> \section{Simulating an inverter}
% or subsections -> \subsection{simulating a particular type of inverter}

The DLX is a RISC architecture, a simplified with a mixture of capabilities from a set of RISC processors such as the Stanford MIPS CPU.
It is a 32-bit load/store architecture. The goal of this project is to build a DLX implementation
from scratch, using VHDL language. There are three main blocks that compose this architecture.

First (not in order of relevance), the Control Unit. The basis for our model had the Hardwired Control Unit. It also manages the pipeline in case of hazards.
The Datapath is another important block of the DLX. Some internal blocks, like the Pentium 4 Adder were previously created during the course laboratories
and were used for this final project. Another component is the T2 Shifter, which was shown in the course lectures and we decided to implement it.
The Hazard Unit is another component that works alongside the Control Unit, to detect the different types of hazards that could occur during the
execution of instructions in the pipeline. Later in this paper, all of its features are explained.
After finishing the design phase, all the components were tested exhaustively using a bottom-up approach.
Each component was singularly tested to verify its correct behavior, using VHDL testbenches.
In the next testing phase, the entire CPU was tested using assembly programs and Modelsim environment to verify the waveforms.
The next step was the synthesis phase, using Synopsys, which forms a Logic Synthesis of the CPU along with some optimizations. Thanks to this step, the
step the maximum frequency of the implemented processor is 667 MegaHertz.
The last step was the physical design phase, using Innovus.
All the process steps will be described in this report, along with all the features provided by this DLX implementation.


