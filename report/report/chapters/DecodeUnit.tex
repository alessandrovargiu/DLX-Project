%%%%%%%%%%%%%%%%%%%%%%%%%%%%%%%%%%%%%%%%%%%%%%%%%%%%
% This will help you in writing your homebook
% Remember that the character % is a comment in latex
%
% chapter 4
\chapter{Decode Unit}
\label{DecodeUnit}

In this part of the datapath the instruction that was previously fetched is broken down and given to several components.
This part of the datapath contains more circuitry with respect to the fetch unit:

\begin{enumerate}
    \item Sign Extender Module and 4to1 multiplexer;
    \item Register File;
    \item Pipe Registers for subsequent execution phase;
    \end{enumerate}

\section {Sign Extender Module and four to one multiplexer}

    The sign extender module is designed with the final goal of interpreting correctly the immidate bit field stored in an immidiate type or jump type instructions.
    In fact, as the implemented DLX gives the possibility to distingush operations for unsigned and signed operations, this combinational block 
    gives the possibility to extend the most significant bit of the immediate field of the immediate type instruction (sixteenth least significant bit) to the the other sixteen bits. 
    The same is done does  for the immediate field of the Jump-Type instructions (this time containing a twentysix bit width instead of sixteen).
    The reason for this choice is because the DLX wires all have thirtytwo bit width, hence we must extend the original immidiate value of the instruction accodingly.
    Furthermore, there are four possible possibilities of interest that could be presented: 
    \begin{enumerate}

        \item an Immidiate-Type instruction interpreting the immediate field as signed: in this case depending on the Most significant bit of the immediate
        bit field, the sixteen additional most signinfiant bits will be either zero, or one. in this scenario we are extending a sixteen bit immediate to a thritytwo bit value
        
        \item an Immidiate-Type instruction interpreting the immediate field as unsigned: in this case regardless of the Most significant bit of the immediate
        bit field, the sixteen additional most signinfiant bits will be zero. in this scenario we are extending a sixteen bit immediate to a thritytwo bit value
     
        \item a Jump-type instruction interpreting the immediate field as as signed: in this case depending on the Most significant bit of the immediate
        bit field, the six additional most signinfiant bits will be either zero, or one. in this scenario we are extending a twenty six bit immediate to a thritytwo bit value

        \item a Jump-type instruction interpreting the immediate field as unsigned: This scenario is not implemented in the datapth as the jump instructions always consider
        a signed value of the immediate field. However for a future improvement of the DLX it could be used for a type of jump instruction that can only jump forward.

        \end{enumerate}

    All these scenarios are performed in parallel in the decode unit, and thanks to a four to one multiplexer, the correct specific scenario is selected for obtaining
    the appropriate extension. Furthermore, the selection bits of the multplexer individuating the correct extension scenario come from two control bits from the control unit.

\section {the Register file}

    
    