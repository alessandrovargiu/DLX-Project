\chapter{Synthesis}
\label{Synthesis}

The synthesis process was done after the Simulation phase, in which the design was simulated to ensure correct behavior of the different units of the CPU, and the CPU itself.
The usage of scripts to automate the synthesis steps is crucial, since it allows us to produce the design netlists and reports saving a lot of time.
% describe scripts
The script synthesis.tcl was created to automate the synthesis and optimization of the CPU under timing and power constraints.

Various steps are performed in this phase, from generating detailed models of the design, with timing, area and power reports, to implement optimizations
with regard to timing constraints, for example.

The whole process is composed of multiple steps:

\begin{enumerate}
    \item First, all vhd files needed for the DLX synthesis are analyzed (testbenches are not used in synthesis)
    \item The design is then elaborated and high level modeles are produced. At this point, at this point we don't have many information about area, timing and power consumption.
    \item The wire model is set and the clock signal is created
    \item The design is compiled and the netlists are generated, ready to be analyzed by a physical design tool.
    \item The timing, area and power reports are saved.
\end{enumerate}

The timing report is especially important to see if the chosen clock period value is suitable for this design.
The report shows the critical path(s) of the design and tells if the \textbf{slack} is met.

This DLX design contains different optimizations with regard to combinational circuits.
The P4 Adder, created during the course labs, was used in this design. The T2 shifter is another addition to the design, which provides speed advantage to shifting operations.

Since the finding of the right clock period was a sort of trial and error process, the timing report was used to see if some other optimizations were possible.
The first timing report shown a critical path in a RCA (Ripple Carry Adder) module, which was used in the incrementing of the program counter.
The RCA was implemented using a structural design, which provided a lot of delay, due to the carry propagation in the circuit.
We switched to a behavioral implementation, letting the synthesis libraries do the job for the optimal netlist. The timing performance was substantially better and we managed
to get a clock period of 1.50 ns for the whole DLX.
At this point the shown critical path is a series of gates which belong to the hazard unit. Since the provided Hazard Unit has a high level behavioral implementation, it was assumed that we could not
optimize very much at a high level of abstraction.

Analyzing the other reports, area and power information can be provided.
The DLX has a total cell area of 19141.89, which is divided between combinational area (8752.19) and non-combinational area (10389.69).
The total dynamic power is 818.39 mW, with a leakage power of 399.45 uW
