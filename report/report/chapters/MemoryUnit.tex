\chapter{Memory Stage}
\label{Memory Unit}
Memory Unit is the fourth stage of the pipeline. It is composed by the Data Memory and the pipeline registers. There are five different registers:
\begin{enumerate} 
    \item Three out of five are used to pipeline the destination address, the npc and the instruction.
    \item One is used to pipeline the result of the ALU from the execution stage.
    \item The last one is used to take the output of the Dram in the memory stage in case there is the need to write it to the register file (load instruction).
\end{enumerate} 
\section{Dram}
The Dram contains the data for load/store instructions and the values available at the start of the execution are initialized during the reset phase thanks to "DMEM\_init\_file.mem". 
This way, initial values are loaded into the DRAM at startup which can be a helpful addition during the testing of the store and load instructions in Modelsim
The correct instruction is executed based on RW signal(read/write signal) that it is sent by the Control Unit.
\subsection{Read and Write processes}
During the process,  as inputs arrive an unique address, the data pipelined from the execution unit (RegB), RW and EN from the Control Unit. The read operation starts 
when EN='1' and RW='1' (load instruction) instead for write operation when EN='1' and RW='0' (store instruction), both synchronous with the falling edge of the clock.
The output ,in case of the read operation, is sent to the pipeline register LMDreg in memory unit to be pipelined to writeback.